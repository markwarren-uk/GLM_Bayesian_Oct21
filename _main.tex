% Options for packages loaded elsewhere
\PassOptionsToPackage{unicode}{hyperref}
\PassOptionsToPackage{hyphens}{url}
%
\documentclass[
]{book}
\title{\emph{Bayesian GLMs in R for Ecology}}
\author{Mark Warren \& Carl Smith}
\date{November 2021}

\usepackage{amsmath,amssymb}
\usepackage{lmodern}
\usepackage{iftex}
\ifPDFTeX
  \usepackage[T1]{fontenc}
  \usepackage[utf8]{inputenc}
  \usepackage{textcomp} % provide euro and other symbols
\else % if luatex or xetex
  \usepackage{unicode-math}
  \defaultfontfeatures{Scale=MatchLowercase}
  \defaultfontfeatures[\rmfamily]{Ligatures=TeX,Scale=1}
  \setmainfont[]{Gill Sans}
\fi
% Use upquote if available, for straight quotes in verbatim environments
\IfFileExists{upquote.sty}{\usepackage{upquote}}{}
\IfFileExists{microtype.sty}{% use microtype if available
  \usepackage[]{microtype}
  \UseMicrotypeSet[protrusion]{basicmath} % disable protrusion for tt fonts
}{}
\makeatletter
\@ifundefined{KOMAClassName}{% if non-KOMA class
  \IfFileExists{parskip.sty}{%
    \usepackage{parskip}
  }{% else
    \setlength{\parindent}{0pt}
    \setlength{\parskip}{6pt plus 2pt minus 1pt}}
}{% if KOMA class
  \KOMAoptions{parskip=half}}
\makeatother
\usepackage{xcolor}
\IfFileExists{xurl.sty}{\usepackage{xurl}}{} % add URL line breaks if available
\IfFileExists{bookmark.sty}{\usepackage{bookmark}}{\usepackage{hyperref}}
\hypersetup{
  pdftitle={Bayesian GLMs in R for Ecology},
  pdfauthor={Mark Warren \& Carl Smith},
  hidelinks,
  pdfcreator={LaTeX via pandoc}}
\urlstyle{same} % disable monospaced font for URLs
\usepackage[a5paper, margin = 20mm, portrait]{geometry}
\usepackage{longtable,booktabs,array}
\usepackage{calc} % for calculating minipage widths
% Correct order of tables after \paragraph or \subparagraph
\usepackage{etoolbox}
\makeatletter
\patchcmd\longtable{\par}{\if@noskipsec\mbox{}\fi\par}{}{}
\makeatother
% Allow footnotes in longtable head/foot
\IfFileExists{footnotehyper.sty}{\usepackage{footnotehyper}}{\usepackage{footnote}}
\makesavenoteenv{longtable}
\usepackage{graphicx}
\makeatletter
\def\maxwidth{\ifdim\Gin@nat@width>\linewidth\linewidth\else\Gin@nat@width\fi}
\def\maxheight{\ifdim\Gin@nat@height>\textheight\textheight\else\Gin@nat@height\fi}
\makeatother
% Scale images if necessary, so that they will not overflow the page
% margins by default, and it is still possible to overwrite the defaults
% using explicit options in \includegraphics[width, height, ...]{}
\setkeys{Gin}{width=\maxwidth,height=\maxheight,keepaspectratio}
% Set default figure placement to htbp
\makeatletter
\def\fps@figure{htbp}
\makeatother
\setlength{\emergencystretch}{3em} % prevent overfull lines
\providecommand{\tightlist}{%
  \setlength{\itemsep}{0pt}\setlength{\parskip}{0pt}}
\setcounter{secnumdepth}{5}
\newlength{\cslhangindent}
\setlength{\cslhangindent}{1.5em}
\newlength{\csllabelwidth}
\setlength{\csllabelwidth}{3em}
\newlength{\cslentryspacingunit} % times entry-spacing
\setlength{\cslentryspacingunit}{\parskip}
\newenvironment{CSLReferences}[2] % #1 hanging-ident, #2 entry spacing
 {% don't indent paragraphs
  \setlength{\parindent}{0pt}
  % turn on hanging indent if param 1 is 1
  \ifodd #1
  \let\oldpar\par
  \def\par{\hangindent=\cslhangindent\oldpar}
  \fi
  % set entry spacing
  \setlength{\parskip}{#2\cslentryspacingunit}
 }%
 {}
\usepackage{calc}
\newcommand{\CSLBlock}[1]{#1\hfill\break}
\newcommand{\CSLLeftMargin}[1]{\parbox[t]{\csllabelwidth}{#1}}
\newcommand{\CSLRightInline}[1]{\parbox[t]{\linewidth - \csllabelwidth}{#1}\break}
\newcommand{\CSLIndent}[1]{\hspace{\cslhangindent}#1}

\ifLuaTeX
  \usepackage{selnolig}  % disable illegal ligatures
\fi

\begin{document}
\frontmatter
\maketitle

{
\setcounter{tocdepth}{3}
\tableofcontents
}
\mainmatter
\hypertarget{section}{%
\chapter*{}\label{section}}
\addcontentsline{toc}{chapter}{}

\begin{quote}
``\emph{The truth is not for all men, but only for those who seek it.}''

Ayn Rand
\end{quote}

\hypertarget{preface}{%
\chapter*{Preface}\label{preface}}
\addcontentsline{toc}{chapter}{Preface}

Our goal is to produce a set of accessible, inexpensive statistics
guides for undergraduate and post-graduate students that are tailored to
specific fields and that use R. These books present minimal statistical
theory and are intended to allow students to understand the process of
data exploration and model fitting and validation using datasets
comparable to their own and, thereby, encourage the development of
statistical skills.

To obtain the R script and data associated with each book chapter, type
the following into your browser and download the data and script files:

\url{https://www.dropbox.com/s/ofljovoiyp5u6ut/DataScript.zip}

All profits from this book will be donated to the SAS Regimental
Association.

\hypertarget{contributors}{%
\chapter*{Contributors}\label{contributors}}
\addcontentsline{toc}{chapter}{Contributors}

\textbf{Mark Warren}

Environment Agency, Tewkesbury GL20 8JG, UK

email:
\href{mailto:mark.warren@environment-agency.gov.uk}{\nolinkurl{mark.warren@environment-agency.gov.uk}}

\textbf{Carl Smith}

Department of Ecology \& Vertebrate Zoology, University of Łódź, 12/16
Banacha Street, 90-237 Łódź, Poland

and

Institute of Vertebrate Biology, Academy of Sciences of the Czech
Republic, Květná 8, 603 65 Brno, Czech Republic

email:
\href{mailto:carl.smith@biol.uni.lodz.pl}{\nolinkurl{carl.smith@biol.uni.lodz.pl}}

\hypertarget{cover-art}{%
\chapter*{Cover art}\label{cover-art}}
\addcontentsline{toc}{chapter}{Cover art}

The cover art is the work of Laura Andrew (www.lauraandrew.com). Laura
is located in central Lincoln, UK. After studying and working as an
illustrator in London, Laura returned to her roots in Lincolnshire where
she produces her art, and offers courses and workshops to people looking
to learn new skills such as watercolour, oil painting and printmaking.
Much of her art is inspired by the natural world, particularly birds.
Working professionally as both an artist and illustrator Laura sells her
art worldwide and her paintings have been exhibited in galleries locally
and in London.

\hypertarget{intro1}{%
\chapter{Introduction to R and RStudio}\label{intro1}}

In this chapter we will introduce R, which is a programming language for
data analysis and graphics, and RStudio, which is an Integrated
Development Environment (IDE) that allows you to interact more easily
with R. We highly recommend using RStudio as your console for R. As you
become familiar with it you will learn more of its functionality.

The advent of the statistical software package R has contributed
substantially to an improvement in the quality and sophistication of
data analyses performed in a range of scientific fields, including
ecology. While not intuitive to use, R has become the industry standard,
and time invested in learning to master R will be rewarded with an
improved understanding of how to handle and model data. There are
several benefits to using R. First, it is extremely flexible and permits
analysis of almost any type of data. Second, there are extremely
efficient packages that permit the import of various data types, joining
and transforming data, and visualising data. R readily permits the
sharing of code with collaborators or journal reviewers and can be
archived with corresponding datasets for others to use and improve upon.
Finally, R is freely distributed under General Public License for all
major computing platforms (Windows, MacOS and Linux), and under
continuous development by a large community of scientists.

To install the R software on your computer go to:
\url{https://www.r-project.org} and follow the instructions for
downloading the latest version.

To install RStudio go to \url{https://rstudio.com} and download `RStudio
Desktop.'

You will need to install the R software before installing RStudio.

Once both are installed always start any R session by opening RStudio (R
will be opened automatically).

\hypertarget{start}{%
\section{Getting started with R and RStudio}\label{start}}

\hypertarget{basic}{%
\subsection{Basic points}\label{basic}}

\begin{itemize}
\tightlist
\item
  R is command-line driven.
\item
  It requires you to type commands after a command prompt (\textgreater)
  that appears when you open R. After typing a command in the R console
  and pressing `Enter' on your keyboard, the command will run. If your
  command is not complete, R issues a continuation prompt (`+').
\item
  You can also write \emph{script} in the script window, and select a
  \emph{command}, and click the Run button. This R script can be saved.
\item
  Finally, you can also import R script - either saved by you
  previously, or written by someone else.
\item
  R is case sensitive. Take care with spelling and capitalization.
\item
  Commands in R are called `functions' (see Section @ref(functions)).
\item
  The up arrow (\^{}) on your keyboard can be used to bring up previous
  commands that you have typed in the R console.
\item
  The dollar symbol \texttt{\$} is used to select a particular column
  within your data (called a `dataframe') (e.g.~``df\$var1'').
\item
  You can include text in your script that R will not execute by
  including the `hash tag' \# symbol. R ignores the remainder of the
  script line following \#. Using \# enables you to insert comments and
  instructions in your script, or to make modifications to your analysis
  by `hashing out' lines of script.
\end{itemize}

\hypertarget{Rstudio}{%
\subsection{Navigating RStudio}\label{Rstudio}}

The RStudio interface comprises four windows, which by default are
organised as follows:

\begin{enumerate}
\def\labelenumi{\arabic{enumi}.}
\item
  Bottom Left: \emph{Console/Terminal/Jobs} window. For now we will just
  focus on the \emph{Console} tab. Here you can directly enter commands
  after the ``\textgreater{}'' prompt. This is where you will see R
  execute your commands and where results will appear. However you
  cannot save anything written here. Instead, it is more efficient to
  write your commands on the \emph{Source} window (see below) and leave
  this window for results.
\item
  Top Left: \emph{Source} window. Here you can write commands (organised
  as script) that can be edited and saved. If no script has been
  selected the window will appear as ``Untitled.'' It is in the Source
  window that you will do all your work - writing and editing script,
  and pasting in script from other sources. The contents of the entire
  window can be selected (CTRL+A) and then executed using the ``Run''
  command at the top right of the window. Alternatively, you can run
  script one line at a time by placing the cursor anywhere on a line of
  interest and clicking RUN. You can also run a line of script from the
  keyboard using CTRL+ENTER. When a line of script is run in the Source
  window you will see that it is sent to the \emph{Console} window to be
  executed.
\end{enumerate}

If you cannot see this window just open it with:

\emph{File \textgreater\textgreater{} NewFile \textgreater\textgreater{}
RScript}

\begin{enumerate}
\def\labelenumi{\arabic{enumi}.}
\setcounter{enumi}{2}
\item
  Top right: \emph{Environment/History} window. If you select
  Environment you can see the data and values R has in its memory. If
  you select History you can see a history of what has been executed in
  the \emph{Console} window.
\item
  Bottom right: \emph{Files/Plots/Packages/Help/Viewer} window. Here you
  can open, delete, rename files, create folders, view current and
  previous plots, install and load packages or access help.
\end{enumerate}

\hypertarget{RS-settings}{%
\subsection{Basic settings in RStudio}\label{RS-settings}}

The following are our recommendations about how to initially set up
RStudio.

To access the setting menu (once you have opened RStudio):

\emph{Tools \textgreater\textgreater{} Global Options}

Select the `General' tab and deselect the `Restore .RData into workspace
at startup,' and set `Save workspace to .RData on exit' to `Never.'

This procedure ensures that the content of any previous R session is not
stored or reloaded between R sessions, which guarantees that the R
session is `clean' to begin with and does not have unexpected objects or
settings that might interfere with the new code you will input.

Next, within `Global Options' select `Code.' We suggest you go to the
`Execution' section and change the drop-down menu after `Ctrl+Enter
executes:' to `Current line'; as a beginner we recommend that you run
and read R script line-by-line to better understand it.

Finally, still within `Global Options,' select `Appearance.' Here you
can change font style, font size and the editor theme to whatever you
find most pleasing to the eye under different working conditions and
your personal preference. For example, the `Editor theme' of `Vibrant
Ink' works well if your computer screen is in sunlight, whereas `Xcode'
works well under low light conditions.

\hypertarget{Principles}{%
\subsection{Basic principles in R}\label{Principles}}

In Section @ref(basic) we mentioned the term \emph{command}, which is an
instruction given to R and stored/written in a Script, which can be
saved as a file to be shared with others. Commands can be simple
mathematical operations; i.e.~1 + 1, or can be a complex set of
instructions that will execute a full analysis of your data. In the
latter case, R users like us are not expected to work out the different
steps to carry out the analyses (e.g.~a t-test), instead we call/request
a function, which can be thought of as a data analysis method.
Conceptually this is no different to using other software and pressing a
menu button to run a function, except in R you run it via code.
Functions are designed by advanced R users, often called developers, and
one day you will probably write your own functions to help speed up your
analyses, which is another reason why R is so useful. We will routinely
use functions in future sessions, so this concept will become clearer
(see Section @ref(functions)). Some developers compile sets of functions
into packages that are designed for particular types of data (i.e.~data
that contain lots of zeros) or analyses (i.e.~for analysing evolutionary
trees). See Section @ref(functions) for a fuller explanation of
functions and packages.

One of the great attractions about learning R for data analysis is that
thousands of developers are continually working to develop new functions
and packages. Remarkably, they do this work entirely for free. Producers
of commercial statistics packages, such as SPSS and Minitab, employ
their own developers, but fewer than voluntarily contribute to R. The
outcome is that more cutting-edge statistical methods are available to R
users than those reliant on commercially produced statistical software.

\textbf{Objects}

R is referred to as an object-based programming language. Statistical
analyses are based around creating and manipulating objects. Creating
objects is straightforward, thus with the script:

\texttt{a\ \textless{}-\ 1}

We create an object `a' with the value of 1. The symbol `\textless-` is
termed the assignment operator, and it assigns whatever is on its
right-hand side to whatever is on its left-hand side. It is a type of
function (see Section @ref(functions)). The assignment operator keyboard
shortcut is ALT and'-' (Windows) or Option and `-' (Mac). Remember this
shortcut and use it to save typing and time!

We can examine the value of object a by typing its name in the
Editor/Script window and clicking RUN (or CTRL+ENTER on the keyboard):

\texttt{a}

Which will return in the Console window:

1

Objects can be manipulated, thus:

\texttt{a\ +\ 1}

Which returns:

2

Unfortunately, this exciting result has now been lost (to recreate it we
would need to type a + 1 again). Statisticians pride themselves on being
efficient (lazy), so to reliably recreate the same outcome we can make
another object:

\texttt{b\ \textless{}-\ a\ +\ 1}

We have created an object `b' with the value a + 1. We can recall the
value of object b by typing its name in the Editor/Script window (and
CTRL + ENTER):

\texttt{b}

Which will return in the Console window:

2

Objects can contain anything: values (as above), vectors, matrices,
dataframes, lists, graphs, tables, even R script.

The most common data objects in R are \emph{vectors} and
\emph{dataframes}. A vector is a single column in a spreadsheet and will
often be classed as either numeric, character or logical.

Consider two vectors representing the abundance of five fish species
captured in a stretch of the River Pilica in Poland:

\begin{longtable}[]{@{}lc@{}}
\toprule
species & frequency \\
\midrule
\endhead
pike & 40 \\
roach & 99 \\
chub & 31 \\
perch & 35 \\
asp & 0 \\
\bottomrule
\end{longtable}

The vector \emph{species} is a character vector and \emph{frequency} is
a numeric vector.

It is easy to input a vector object in R, just write the following in
the Editor/Script window and run (CTRL + ENTER):

\texttt{freq\ \textless{}-\ c(40,99,31,35,0)}

We have created a vector object `freq.' The `c()' expression is referred
to as the `concatenate' function, combining all the elements in the
parentheses into a vector. Confirm the object contains the correct
values by typing:

\texttt{freq}

Which will return in the Console window:

40, 99, 31, 35, 0

To check the class of vector type:

\texttt{class(freq)}

Which returns:

numeric

We can similarly create an object vector to represent categories:

\texttt{species\ \textless{}-\ c("pike",\ "roach",\ "chub",\ "perch",\ "asp")}

Confirm the object contains the values by typing:

\texttt{species}

Which will return in the Console window:

pike, roach, chub, perch, asp

Another type of object is a \emph{dataframe}, what you might consider as
a `table' of information. You can create dataframes by combining
vectors. So:

\texttt{spec\_freq\ \textless{}-\ data.frame(species,freq)}

Confirm the object contains the values by typing:

\texttt{spec\_freq}

Again, use the \texttt{class()} function to see what type of object
\texttt{spec\_freq} is.

Dataframes are the fundamental objects used within our statistical
analyses. In reality, we usually do not create dataframes by typing data
directly into R when we want to analyse them. Instead, we import them
directly from their source. But first, we need to set up where we will
not only access our data but also our code, outputs and other
documentation associated with particular analyses.

\textbf{A note on tibbles}

Although we have mentioned dataframes, there also exist special types of
dataframe called \emph{tibbles}. Tibbles are dataframes that have been
tweaked to overcome older R behaviours and make life a little easier.
For now simply view tibble as an alias for dataframe and for brevity we
will use the term \emph{dataframe} throughout to cover both. There is a
tibble package, part of the tidyverse set of packages, which we use in
this book. A good starting point to become familiar with
\emph{tidyverse} packages and tidy coding style is the book \emph{R for
Data Science} by Hadley Wickham \& Garrett Grolemund (2016).

\hypertarget{RS-projects}{%
\subsection{Working with RStudio projects}\label{RS-projects}}

When we started using R in 2011, the insightful developers at RStudio
had only just released a beta version of the IDE. Version 1.0 was
released in 2016 and since then the IDE and the `ecosystem' of packages
and add-ins has grown enormously to allow anyone to use R in a more
organised way. Here we want to highlight what we believe is fundamental
to an efficient workflow with the IDE; \emph{projects}.

RStudio projects make it straightforward to divide and manage your work
into multiple contexts, each with their own working directory,
workspace, history, and source documents. Projects are associated with
particular working directories that you set up, basically a folder
somewhere on your computer or network where you store your data. Once a
project is set up and associated with a folder (working directory) your
data, R code and outputs from R will be stored there. Projects set up a
neat and easy to use mini environment for your work and using them is
good practice.

\hypertarget{create-proj}{%
\subsubsection{Creating projects}\label{create-proj}}

RStudio projects are associated with R working directories. You can
create an RStudio project:

\emph{File \textgreater\textgreater{} New Project}

You then have the following options:

\begin{itemize}
\tightlist
\item
  In a brand new directory
\item
  In an existing directory where you already have R code and data
\item
  By cloning a version control (Git or Subversion) repository (this is
  beyond the scope of this book but will feature in our forthcoming book
  \emph{Reproducible Research in R}).
\end{itemize}

When a new project is created RStudio will:

\begin{itemize}
\tightlist
\item
  Create a project file (with an .Rproj extension) within the project
  directory. This file can be used as a shortcut for opening the project
  directly from the filesystem.
\item
  Create a hidden directory (named .Rproj.user) where project-specific
  temporary files (e.g.~auto-saved source documents, window-state, etc.)
  are stored.
\item
  Load the project into RStudio and display its name in the Projects
  toolbar (which is located on the far right side of the main toolbar)
\end{itemize}

\hypertarget{openproj}{%
\subsubsection{Opening and closing projects}\label{openproj}}

There are several ways to open a project but the most common are:

\begin{itemize}
\tightlist
\item
  Using the Open Project command (available from both the Projects menu
  and the Projects toolbar) to browse for and select an existing project
  file (e.g.~MyProject.Rproj).
\item
  Selecting a project from the list of most recently opened projects
  (also available from both the Projects menu and toolbar).
\end{itemize}

When a project is opened within RStudio the following actions are taken:

\begin{itemize}
\tightlist
\item
  A new R session (process) is started
\item
  The .Rprofile file in the project's main directory is sourced by R
\item
  The .Rhistory file in the project's main directory is loaded into the
  RStudio History pane (and used for Console Up/Down arrow command
  history).
\item
  Previously edited source documents are restored into editor tabs
\item
  Other RStudio settings (e.g.~active tabs, splitter positions, etc.)
  are restored to where they were the last time the project was closed.
\end{itemize}

When you are within a project and choose to either Quit, close the
project, or open another project the following actions are taken:

\begin{itemize}
\tightlist
\item
  .RData and/or .Rhistory are written to the project directory (if
  current options indicate they should be)
\item
  The list of open source documents is saved (so it can be restored next
  time the project is opened)
\item
  Your RStudio settings are saved.
\item
  The R session is terminated.
\end{itemize}

\hypertarget{multiproj}{%
\subsubsection{Working with multiple projects}\label{multiproj}}

You can work with more than one RStudio project at a time by simply
opening each project in its own instance of RStudio. The simplest way to
accomplish this is:

\emph{File \textgreater\textgreater{} Open Project in New Session}

You can then navigate to the working directory (or folder) where the
project is saved. Then select the project file (.Rproj) and select open
in the dialogue box. A new RStudio and R session will be available in a
new window. This is helpful when you are starting an analysis that is
similar to one you did a while back, having the older project open means
you can copy/paste similar code into new project scripts.

\hypertarget{import}{%
\subsection{Importing data}\label{import}}

Once you have created a project associated with a directory that
contains data, you can import them into R from a variety of formats,
such as .csv, .txt, .xls, etc. For simplicity, we will work with
tab-delimited files (.txt) and comma-separated files (.csv).

We will start by importing a set of data on Eurasian blue tit
(\emph{Cyanistes caeruleus}) nesting success from Wytham Woods. Wytham
Woods is a tract of ancient woodland in Oxfordshire, UK that has been
managed by the University of Oxford since 1942. It is the most
intensively researched area of woodland in the world and has been used
for pioneering ecological research for decades.

Blue tits are abundant in Wytham Woods and readily use artificial nest
boxes placed by researchers. These data (note that `data' is a plural
word - the singular of `data' is `datum') are a subset of 438 records
collected during the breeding seasons of 2001--2003 in Wytham Woods. The
aim of data collection was to investigate which variables predicted
variation in size of nests. An analysis of the full dataset has been
published previously (\protect\hyperlink{ref-O_Neill_2018}{ONeill et
al., 2018}).

Data for blue tit nests are saved in the tab-delimited file `cyan.txt'
and can be imported into a dataframe in R using the command:

\texttt{cyan\ \textless{}-\ read.table(file\ =\ "cyanistes.txt",\ header\ =\ TRUE,\ dec\ =\ ".")}

After importing the dataframe, select Environment in the
\emph{Environment / History} window (top right) and you will see an
entry for `cyan' showing that R now has the data stored in its memory.

\hypertarget{functions}{%
\subsection{Functions and packages}\label{functions}}

Commands performed on values, vectors, objects, and other structures in
R are executed using \emph{functions}. A function is a type of object.

Every function has the form \texttt{function.name()} with arguments
given inside the brackets. Functions require you to give at least one
argument.

Many functions in R come in \emph{packages} although you can create your
own as we will see in later chapters. Packages are folders containing
all the script that is needed to run particular functions. When you
first install R it comes with a set of default packages (base R) and
whenever you open R or RStudio, some of these are loaded to ensure you
have basic functionality. However, as you develop your statistical and
R-coding skills, you will need to load specific packages to do
particular jobs.

As an example, we can use the function \texttt{describe()} from the
package \emph{psych} to report basic summary statistics for the variable
\texttt{depth} in the \texttt{cyan} dataframe. The package psych is not
part of base R, which means that to use the function \texttt{describe()}
we must first install the package \emph{psych}:

\texttt{install.packages("psych")}

R will go online, download the package from the package repository and
install it to the R library on your computer. Note that to install
packages the package name is wrapped in quotes (``package'')

When a package is installed it is necessary to then \emph{load} it from
the library so that R can access its functions, which is done using the
\texttt{library()} command:

\texttt{library(psych)}

Note that loading a package does not require the quotes around the name
that install.packages() needed. You only need to \emph{install} a
package once. However, after you close R you will have to \emph{load}
any packages that you need again. It is good practice to load all the
packages you will need for your data processing, visualisation, and
analysis at the beginning of your script. You will see this in the
script we provide for analysis in this book.

Now that the package psych is installed and loaded, we can use the
function \texttt{describe()} to summarise the variable \texttt{depth} in
the \texttt{cyan} dataframe. The variable \texttt{depth} is an estimate
of the fraction of each nest box that was filled by a blue tit nest.
Note that the \texttt{\$} symbol is used to select the column
\texttt{depth} in the \texttt{cyan} dataframe.

\texttt{describe(cyan\$depth,\ skew\ =\ FALSE)}

\begin{longtable}[]{@{}ccccccccc@{}}
\toprule
& vars & n & mean & sd & min & max & range & se \\
\midrule
\endhead
X1 & 1 & 438 & 0.33 & 0.1 & 0.17 & 0.75 & 0.58 & 0 \\
\bottomrule
\end{longtable}

Which gives us the number of summarised variables (\texttt{vars}), the
sample size (\texttt{n}), mean of the variable (\texttt{mean}), standard
deviation (\texttt{sd}), minimum value (\texttt{min}), maximum value
(\texttt{max}), range of the variable (\texttt{range}), and standard
error of the mean (\texttt{se}).

\hypertarget{introbayesian}{%
\chapter{Introduction to Bayesian inference}\label{introbayesian}}

Placeholder

\hypertarget{freqbayes}{%
\section{The differences between Bayesian and frequentist
approaches}\label{freqbayes}}

\hypertarget{freq}{%
\subsection{Frequentist approach}\label{freq}}

\hypertarget{bayesian}{%
\subsection{Bayesian approach}\label{bayesian}}

\hypertarget{theorum}{%
\subsection{Bayes' theorum}\label{theorum}}

\hypertarget{which}{%
\subsection{A frequentist or Bayesian framework?}\label{which}}

\hypertarget{fit-glms}{%
\section{Fitting Bayesian GLMs}\label{fit-glms}}

\hypertarget{fit-steps}{%
\subsection{Steps in fitting a Bayesian GLM}\label{fit-steps}}

\hypertarget{intro-priors}{%
\section{Priors}\label{intro-priors}}

\hypertarget{flat-priors}{%
\subsection{Non-informative priors}\label{flat-priors}}

\hypertarget{weak-priors}{%
\subsection{Weakly-informative priors}\label{weak-priors}}

\hypertarget{inform-priors}{%
\subsection{Informative priors}\label{inform-priors}}

\hypertarget{conj-priors}{%
\subsection{Conjugate priors}\label{conj-priors}}

\hypertarget{post-dist}{%
\section{The posterior distribution}\label{post-dist}}

\hypertarget{comp-methods}{%
\section{Bayesian computational methods}\label{comp-methods}}

\hypertarget{mcmc}{%
\subsection{Markov chain Monte Carlo sampling (MCMC)}\label{mcmc}}

\hypertarget{inla}{%
\subsection{Numerical approximation}\label{inla}}

\hypertarget{bayes-pros}{%
\section{The advantages of Bayesian inference}\label{bayes-pros}}

\hypertarget{bayes-critics}{%
\section{Criticism of Bayesian inference}\label{bayes-critics}}

\hypertarget{bayes-concl}{%
\section{Conclusions}\label{bayes-concl}}

\hypertarget{data-exploration}{%
\chapter{Data exploration}\label{data-exploration}}

Placeholder

\hypertarget{six-step-data-exploration-protocol}{%
\section{Six-step data exploration
protocol}\label{six-step-data-exploration-protocol}}

\hypertarget{outliers}{%
\subsection{Outliers}\label{outliers}}

\hypertarget{normality-and-homogeneity-of-the-dependent-variable}{%
\subsection{Normality and homogeneity of the dependent
variable}\label{normality-and-homogeneity-of-the-dependent-variable}}

\hypertarget{lots-of-zeros-in-the-response-variable}{%
\subsection{Lots of zeros in the response
variable}\label{lots-of-zeros-in-the-response-variable}}

\hypertarget{multicollinearity-among-covariates}{%
\subsection{Multicollinearity among
covariates}\label{multicollinearity-among-covariates}}

\hypertarget{relationships-among-dependent-and-independent-variables}{%
\subsection{Relationships among dependent and independent
variables}\label{relationships-among-dependent-and-independent-variables}}

\hypertarget{independence}{%
\subsection{Independence of response variable}\label{independence}}

\hypertarget{results-of-data-exploration}{%
\section{Results of data
exploration}\label{results-of-data-exploration}}

\hypertarget{conclusions}{%
\section{Conclusions}\label{conclusions}}

\hypertarget{gen-model}{%
\chapter{Bayesian GLM}\label{gen-model}}

Placeholder

\hypertarget{bitterling}{%
\section{European bitterling territoriality}\label{bitterling}}

\hypertarget{glm-steps}{%
\section{Steps in fitting a Bayesian GLM}\label{glm-steps}}

\hypertarget{state-the-question}{%
\subsection{State the question}\label{state-the-question}}

\hypertarget{data-exploration-1}{%
\subsection{Data exploration}\label{data-exploration-1}}

\hypertarget{outliers-1}{%
\subsubsection{Outliers}\label{outliers-1}}

\hypertarget{normality-and-homogeneity-of-the-dependent-variable-1}{%
\subsubsection{Normality and homogeneity of the dependent
variable}\label{normality-and-homogeneity-of-the-dependent-variable-1}}

\hypertarget{balance-of-categorical-variables}{%
\subsubsection{Balance of categorical
variables}\label{balance-of-categorical-variables}}

\hypertarget{multicollinearity-among-covariates-1}{%
\subsubsection{Multicollinearity among
covariates}\label{multicollinearity-among-covariates-1}}

\hypertarget{zeros-in-the-response-variable}{%
\subsubsection{Zeros in the response
variable}\label{zeros-in-the-response-variable}}

\hypertarget{relationships-among-dependent-and-independent-variables-1}{%
\subsubsection{Relationships among dependent and independent
variables}\label{relationships-among-dependent-and-independent-variables-1}}

\hypertarget{independence-of-response-variable}{%
\subsubsection{Independence of response
variable}\label{independence-of-response-variable}}

\hypertarget{selection-of-a-statistical-model}{%
\subsection{Selection of a statistical
model}\label{selection-of-a-statistical-model}}

\hypertarget{specification-of-priors}{%
\subsection{Specification of priors}\label{specification-of-priors}}

\hypertarget{pilot}{%
\subsubsection{Pilot study}\label{pilot}}

\hypertarget{frequentist-linear-model}{%
\subsubsection{Frequentist linear
model}\label{frequentist-linear-model}}

\hypertarget{priors-on-the-fixed-effects}{%
\subsubsection{Priors on the fixed
effects}\label{priors-on-the-fixed-effects}}

\hypertarget{priors-on-the-hyperparameter}{%
\subsubsection{Priors on the
hyperparameter}\label{priors-on-the-hyperparameter}}

\hypertarget{fit-the-model}{%
\subsection{Fit the model}\label{fit-the-model}}

\hypertarget{obtain-the-posterior-distribution}{%
\subsection{Obtain the posterior
distribution}\label{obtain-the-posterior-distribution}}

\hypertarget{model-with-default-priors}{%
\subsubsection{Model with default
priors}\label{model-with-default-priors}}

\hypertarget{fixed-effects}{%
\paragraph{Fixed effects}\label{fixed-effects}}

\hypertarget{hyperparameter}{%
\paragraph{Hyperparameter}\label{hyperparameter}}

\hypertarget{model-with-informative-priors}{%
\subsubsection{Model with informative
priors}\label{model-with-informative-priors}}

\hypertarget{fixed-effects-1}{%
\paragraph{Fixed effects}\label{fixed-effects-1}}

\hypertarget{hyperparameter-1}{%
\paragraph{Hyperparameter}\label{hyperparameter-1}}

\hypertarget{comparison-with-frequentist-gaussian-glm}{%
\subsubsection{Comparison with frequentist Gaussian
GLM}\label{comparison-with-frequentist-gaussian-glm}}

\hypertarget{conduct-model-checks}{%
\subsection{Conduct model checks}\label{conduct-model-checks}}

\hypertarget{model-selection-using-the-deviance-information-criterion-dic}{%
\subsubsection{Model selection using the Deviance Information Criterion
(DIC)}\label{model-selection-using-the-deviance-information-criterion-dic}}

\hypertarget{posterior-predictive-checks}{%
\subsubsection{Posterior predictive
checks}\label{posterior-predictive-checks}}

\hypertarget{cross-validation-model-checking}{%
\subsubsection{Cross-validation model
checking}\label{cross-validation-model-checking}}

\hypertarget{bayesian-residuals-analysis}{%
\subsubsection{Bayesian residuals
analysis}\label{bayesian-residuals-analysis}}

\hypertarget{prior-sensitivity-analysis}{%
\subsubsection{Prior sensitivity
analysis}\label{prior-sensitivity-analysis}}

\hypertarget{conclusions-from-model-checks}{%
\subsubsection{Conclusions from model
checks}\label{conclusions-from-model-checks}}

\hypertarget{interpret-and-present-model-output}{%
\subsection{Interpret and present model
output}\label{interpret-and-present-model-output}}

\hypertarget{visualise-the-results}{%
\subsection{Visualise the results}\label{visualise-the-results}}

\hypertarget{conclusions-1}{%
\section{Conclusions}\label{conclusions-1}}

\hypertarget{pois-glm}{%
\chapter{Bayesian Poisson GLM}\label{pois-glm}}

Placeholder

\hypertarget{ga-plate}{%
\section{Stickleback lateral plate number}\label{ga-plate}}

\hypertarget{pois-glm-steps}{%
\section{Steps in fitting a Bayesian GLM}\label{pois-glm-steps}}

\hypertarget{ga-question}{%
\subsection{State the question}\label{ga-question}}

\hypertarget{ga-eda}{%
\subsection{Data exploration}\label{ga-eda}}

\hypertarget{ga-outliers}{%
\subsubsection{Outliers}\label{ga-outliers}}

\hypertarget{pois-dist}{%
\subsubsection{Distribution of the dependent variable}\label{pois-dist}}

\hypertarget{pois-balance}{%
\subsubsection{Balance of categorical variables}\label{pois-balance}}

\hypertarget{pois-collin}{%
\subsubsection{Multicollinearity among covariates}\label{pois-collin}}

\hypertarget{pois-zeros}{%
\subsubsection{Zeros in the response variable}\label{pois-zeros}}

\hypertarget{pois-rels}{%
\subsubsection{Relationships among dependent and independent
variables}\label{pois-rels}}

\hypertarget{pois-indep}{%
\subsubsection{Independence of response variable}\label{pois-indep}}

\hypertarget{pois-select}{%
\subsection{Selection of a statistical model}\label{pois-select}}

\hypertarget{pois-prior-spec}{%
\subsection{Specification of priors}\label{pois-prior-spec}}

\hypertarget{existing-data}{%
\subsubsection{Existing data}\label{existing-data}}

\hypertarget{pois-priors-fixed}{%
\subsubsection{Priors on the fixed effects}\label{pois-priors-fixed}}

\hypertarget{pois-fit-models}{%
\subsection{Fit the models}\label{pois-fit-models}}

\hypertarget{obtain-the-posterior-distribution-1}{%
\subsection{Obtain the posterior
distribution}\label{obtain-the-posterior-distribution-1}}

\hypertarget{pois-def-priors}{%
\subsubsection{Model with default priors}\label{pois-def-priors}}

\hypertarget{pois-inf-priors}{%
\subsubsection{Model with informative priors}\label{pois-inf-priors}}

\hypertarget{fixed-effects-2}{%
\subsubsection{Fixed effects}\label{fixed-effects-2}}

\hypertarget{pois-freq-comp}{%
\subsubsection{Comparison with frequentist Gaussian
GLM}\label{pois-freq-comp}}

\hypertarget{conduct-model-checks-1}{%
\subsection{Conduct model checks}\label{conduct-model-checks-1}}

\hypertarget{pois-dic}{%
\subsubsection{Model selection using the Deviance Information Criterion
(DIC)}\label{pois-dic}}

\hypertarget{pois-disp}{%
\subsubsection{Dispersion}\label{pois-disp}}

\hypertarget{pois-sim}{%
\paragraph{Apply Poisson GLM in INLA}\label{pois-sim}}

\hypertarget{simulate-regression-parameters-from-the-posterior-distribution}{%
\paragraph{Simulate regression parameters from the posterior
distribution}\label{simulate-regression-parameters-from-the-posterior-distribution}}

\hypertarget{calculate-predicted-values}{%
\paragraph{Calculate predicted
values}\label{calculate-predicted-values}}

\hypertarget{simulate-count-data-using-rpois}{%
\paragraph{\texorpdfstring{Simulate count data using
\texttt{rpois}}{Simulate count data using rpois}}\label{simulate-count-data-using-rpois}}

\hypertarget{calculate-summary-statistic}{%
\paragraph{Calculate summary
statistic}\label{calculate-summary-statistic}}

\hypertarget{repeat-simulation}{%
\paragraph{Repeat simulation}\label{repeat-simulation}}

\hypertarget{compare-dispersion-of-simulated-and-observed-data}{%
\paragraph{Compare dispersion of simulated and observed
data}\label{compare-dispersion-of-simulated-and-observed-data}}

\hypertarget{pois-ppc}{%
\subsubsection{Posterior predictive checks}\label{pois-ppc}}

\hypertarget{pois-cv}{%
\subsubsection{Cross-validation model checking}\label{pois-cv}}

\hypertarget{pois-resids}{%
\subsubsection{Bayesian residuals analysis}\label{pois-resids}}

\hypertarget{pois-sens}{%
\subsubsection{Prior sensitivity analysis}\label{pois-sens}}

\hypertarget{conclusions-from-model-checks-1}{%
\subsubsection{Conclusions from model
checks}\label{conclusions-from-model-checks-1}}

\hypertarget{pois-present}{%
\subsection{Interpret and present model output}\label{pois-present}}

\hypertarget{visualise-the-results-1}{%
\subsection{Visualise the results}\label{visualise-the-results-1}}

\hypertarget{conclusions-2}{%
\section{Conclusions}\label{conclusions-2}}

\hypertarget{nb-glm}{%
\chapter{Bayesian negative binomial GLM}\label{nb-glm}}

Placeholder

\hypertarget{coral-abundance}{%
\section{Coral abundance}\label{coral-abundance}}

\hypertarget{nb-glm-steps}{%
\section{Steps in fitting a Bayesian GLM}\label{nb-glm-steps}}

\hypertarget{coral-question}{%
\subsection{State the question}\label{coral-question}}

\hypertarget{coral-eda}{%
\subsection{Data exploration}\label{coral-eda}}

\hypertarget{coral-outliers}{%
\subsubsection{Outliers}\label{coral-outliers}}

\hypertarget{nb-dist}{%
\subsubsection{Distribution of the dependent variable}\label{nb-dist}}

\hypertarget{nb-balance}{%
\subsubsection{Balance of categorical variables}\label{nb-balance}}

\hypertarget{nb-collin}{%
\subsubsection{Multicollinearity among covariates}\label{nb-collin}}

\hypertarget{nb-zeros}{%
\subsubsection{Zeros in the response variable}\label{nb-zeros}}

\hypertarget{nb-rels}{%
\subsubsection{Relationships among dependent and independent
variables}\label{nb-rels}}

\hypertarget{nb-indep}{%
\subsubsection{Independence of response variable}\label{nb-indep}}

\hypertarget{nb-select}{%
\subsection{Selection of a statistical model}\label{nb-select}}

\hypertarget{nb-prior-spec}{%
\subsection{Specification of priors}\label{nb-prior-spec}}

\hypertarget{previous-study}{%
\subsubsection{Previous study}\label{previous-study}}

\hypertarget{nb-priors-fixed}{%
\subsubsection{Priors on the fixed effects}\label{nb-priors-fixed}}

\hypertarget{nb-fit-models}{%
\subsection{Fit the models}\label{nb-fit-models}}

\hypertarget{nb-post-dist}{%
\subsection{Obtain the posterior distribution}\label{nb-post-dist}}

\hypertarget{nb-def-priors}{%
\subsubsection{Model with default priors}\label{nb-def-priors}}

\hypertarget{nb-inf-priors}{%
\subsubsection{Model with informative priors}\label{nb-inf-priors}}

\hypertarget{nb-prior-comp}{%
\subsubsection{Comparison of models with uninformative and informative
priors}\label{nb-prior-comp}}

\hypertarget{nb-freq-comp}{%
\subsubsection{Comparison with frequentist Poisson
GLM}\label{nb-freq-comp}}

\hypertarget{conduct-model-checks-2}{%
\subsection{Conduct model checks}\label{conduct-model-checks-2}}

\hypertarget{nb-dic}{%
\subsubsection{Model selection using the Deviance Information Criterion
(DIC)}\label{nb-dic}}

\hypertarget{nb-disp}{%
\subsubsection{Dispersion}\label{nb-disp}}

\hypertarget{nbglm-fit}{%
\subsubsection{Bayesian negative binomial GLM}\label{nbglm-fit}}

\hypertarget{nbglm-ppc}{%
\subsubsection{Posterior predictive checks}\label{nbglm-ppc}}

\hypertarget{nbglm-cv}{%
\subsubsection{Cross-validation model checking}\label{nbglm-cv}}

\hypertarget{nbglm-resids}{%
\subsubsection{Bayesian residuals analysis}\label{nbglm-resids}}

\hypertarget{nb-sens}{%
\subsubsection{Prior sensitivity analysis}\label{nb-sens}}

\hypertarget{nb-checkconc}{%
\subsubsection{Conclusions from model checks}\label{nb-checkconc}}

\hypertarget{nb-present}{%
\subsection{Interpret and present model output}\label{nb-present}}

\hypertarget{visualise-the-results-2}{%
\subsection{Visualise the results}\label{visualise-the-results-2}}

\hypertarget{conclusions-3}{%
\section{Conclusions}\label{conclusions-3}}

\hypertarget{bern-glm}{%
\chapter{Bayesian Bernoulli GLM}\label{bern-glm}}

Placeholder

\hypertarget{bern-cc}{%
\section{Common cuckoo parasitism of great reed warbler
nests}\label{bern-cc}}

\hypertarget{bern-glm-steps}{%
\section{Steps in fitting a Bayesian GLM}\label{bern-glm-steps}}

\hypertarget{cc-question}{%
\subsection{State the question}\label{cc-question}}

\hypertarget{cc-eda}{%
\subsection{Data exploration}\label{cc-eda}}

\hypertarget{cc-outliers}{%
\subsubsection{Outliers}\label{cc-outliers}}

\hypertarget{bern-dist}{%
\subsubsection{Distribution of the dependent variable}\label{bern-dist}}

\hypertarget{bern-balance}{%
\subsubsection{Balance of categorical variables}\label{bern-balance}}

\hypertarget{bern-collin}{%
\subsubsection{Multicollinearity among covariates}\label{bern-collin}}

\hypertarget{bern-zeros}{%
\subsubsection{Zeros in the response variable}\label{bern-zeros}}

\hypertarget{bern-rels}{%
\subsubsection{Relationships among dependent and independent
variables}\label{bern-rels}}

\hypertarget{bern-depend}{%
\subsubsection{Independence of response variable}\label{bern-depend}}

\hypertarget{bern-select}{%
\subsection{Selection of a statistical model}\label{bern-select}}

\hypertarget{bern-prior-spec}{%
\subsection{Specification of priors}\label{bern-prior-spec}}

\hypertarget{pilot-study}{%
\subsubsection{Pilot study}\label{pilot-study}}

\hypertarget{model-pilot-data}{%
\subsubsection{Model pilot data}\label{model-pilot-data}}

\hypertarget{bern-priors-fixed}{%
\subsubsection{Priors on the fixed effects}\label{bern-priors-fixed}}

\hypertarget{bern-fit-models}{%
\subsection{Fit the models}\label{bern-fit-models}}

\hypertarget{bern-post-dist}{%
\subsection{Obtain the posterior distribution}\label{bern-post-dist}}

\hypertarget{model-with-default-priors-1}{%
\subsubsection{Model with default
priors}\label{model-with-default-priors-1}}

\hypertarget{bern-inf-priors}{%
\subsubsection{Model with informative priors}\label{bern-inf-priors}}

\hypertarget{bern-prior-comp}{%
\subsubsection{Comparison of models with uninformative and informative
priors}\label{bern-prior-comp}}

\hypertarget{bern-freq-comp}{%
\subsubsection{Comparison with frequentist Bernoulli
GLM}\label{bern-freq-comp}}

\hypertarget{conduct-model-checks-3}{%
\subsection{Conduct model checks}\label{conduct-model-checks-3}}

\hypertarget{bern-dic}{%
\subsubsection{Model selection using the Deviance Information Criterion
(DIC)}\label{bern-dic}}

\hypertarget{bern-ppc}{%
\subsubsection{Posterior predictive checks}\label{bern-ppc}}

\hypertarget{bern-resids}{%
\subsubsection{Bayesian residuals analysis}\label{bern-resids}}

\hypertarget{bern-sens}{%
\subsubsection{Prior sensitivity analysis}\label{bern-sens}}

\hypertarget{bern-checkconc}{%
\subsubsection{Conclusions from model checks}\label{bern-checkconc}}

\hypertarget{bern-present}{%
\subsection{Interpret and present model output}\label{bern-present}}

\hypertarget{visualise-the-results-3}{%
\subsection{Visualise the results}\label{visualise-the-results-3}}

\hypertarget{conclusions-4}{%
\section{Conclusions}\label{conclusions-4}}

\hypertarget{gamma-glm}{%
\chapter{Bayesian gamma GLM}\label{gamma-glm}}

Placeholder

\hypertarget{common-seal-dive-duration}{%
\section{Common seal dive duration}\label{common-seal-dive-duration}}

\hypertarget{gamma-glm-steps}{%
\section{Steps in fitting a Bayesian GLM}\label{gamma-glm-steps}}

\hypertarget{seal-question}{%
\subsection{State the question}\label{seal-question}}

\hypertarget{gamma-eda}{%
\subsection{Data exploration}\label{gamma-eda}}

\hypertarget{outliers-2}{%
\subsubsection{Outliers}\label{outliers-2}}

\hypertarget{gamma-dist}{%
\subsubsection{Distribution of the dependent
variable}\label{gamma-dist}}

\hypertarget{gamma-balance}{%
\subsubsection{Balance of categorical variables}\label{gamma-balance}}

\hypertarget{gamma-collin}{%
\subsubsection{Multicollinearity among covariates}\label{gamma-collin}}

\hypertarget{gamma-zeros}{%
\subsubsection{Zeros in the response variable}\label{gamma-zeros}}

\hypertarget{gamma-rels}{%
\subsubsection{Relationships among dependent and independent
variables}\label{gamma-rels}}

\hypertarget{gamma-indep}{%
\subsubsection{Independence of response variable}\label{gamma-indep}}

\hypertarget{gamma-select}{%
\subsection{Selection of a statistical model}\label{gamma-select}}

\hypertarget{gamma-prior-spec}{%
\subsection{Specification of priors}\label{gamma-prior-spec}}

\hypertarget{gamma-priors-fixed}{%
\subsubsection{Priors on the fixed effects}\label{gamma-priors-fixed}}

\hypertarget{gamma-hyper}{%
\subsubsection{Priors on the hyperparameter}\label{gamma-hyper}}

\hypertarget{gamma-fit-model}{%
\subsection{Fit the model}\label{gamma-fit-model}}

\hypertarget{gamma-post-dist}{%
\subsection{Obtain the posterior distribution}\label{gamma-post-dist}}

\hypertarget{gamma-def-priors}{%
\subsubsection{Model with default priors}\label{gamma-def-priors}}

\hypertarget{gamma-inf-priors}{%
\subsubsection{Model with informative priors}\label{gamma-inf-priors}}

\hypertarget{gamma-prior-comp}{%
\subsubsection{Comparison of models with uninformative and informative
priors}\label{gamma-prior-comp}}

\hypertarget{gamma-freq-comp}{%
\subsubsection{Comparison with frequentist gamma
GLM}\label{gamma-freq-comp}}

\hypertarget{conduct-model-checks-4}{%
\subsection{Conduct model checks}\label{conduct-model-checks-4}}

\hypertarget{gamma-dic}{%
\subsubsection{Model selection using the Deviance Information Criterion
(DIC)}\label{gamma-dic}}

\hypertarget{gamma-ppc}{%
\subsubsection{Posterior predictive checks}\label{gamma-ppc}}

\hypertarget{gamma-cv}{%
\subsubsection{Cross-validation model checking}\label{gamma-cv}}

\hypertarget{gamma-resids}{%
\subsubsection{Bayesian residuals analysis}\label{gamma-resids}}

\hypertarget{gamma-sens}{%
\subsubsection{Prior sensitivity analysis}\label{gamma-sens}}

\hypertarget{gamma-checkconc}{%
\subsubsection{Conclusions from model checks}\label{gamma-checkconc}}

\hypertarget{gamma-present}{%
\subsection{Interpret and present model output}\label{gamma-present}}

\hypertarget{visualise-the-results-4}{%
\subsection{Visualise the results}\label{visualise-the-results-4}}

\hypertarget{conclusions-5}{%
\section{Conclusions}\label{conclusions-5}}

\hypertarget{implementing-and-assessing-bayesian-glms}{%
\chapter{Implementing and assessing Bayesian
GLMs}\label{implementing-and-assessing-bayesian-glms}}

Placeholder

\hypertarget{prior-information}{%
\section{Prior information}\label{prior-information}}

\hypertarget{the-results-of-previous-research}{%
\subsection{The results of previous
research}\label{the-results-of-previous-research}}

\hypertarget{logical-considerations}{%
\subsection{Logical considerations}\label{logical-considerations}}

\hypertarget{expert-knowledge}{%
\subsection{Expert knowledge}\label{expert-knowledge}}

\hypertarget{pilot-data}{%
\subsection{Pilot data}\label{pilot-data}}

\hypertarget{presenting-the-results-of-a-bayesian-glm}{%
\section{Presenting the results of a Bayesian
GLM}\label{presenting-the-results-of-a-bayesian-glm}}

\hypertarget{reviewing-bayesian-glms}{%
\section{Reviewing Bayesian GLMs}\label{reviewing-bayesian-glms}}

\hypertarget{misuse}{%
\section{Misuse of Bayesian inference}\label{misuse}}

\hypertarget{conclusions-6}{%
\section{Conclusions}\label{conclusions-6}}

\hypertarget{coda}{%
\chapter{Coda}\label{coda}}

How can we use new data to change what we currently believe? As
ecologists we often make decisions in the face of uncertainty and
incomplete information. Bayesian inference offers a framework for
incrementally accruing scientific knowledge by explicitly building on
the conclusions of previous knowledge.

However, despite the attraction in using Bayesian inference to tackle
ecological questions, there are many pitfalls to its implementation.
Sovereign against many of these problems is transparency; clearly
reporting how priors were obtained, why they are specified as they are,
careful description of their impacts, and presentation of sensitivity
analyses. Ultimately, Bayesian methods do not offer a panacea, but they
are a valuable tool for the ecologist that encourages full use of the
available data - whatever form those data take.

We hope this book is useful in extending your understanding of Bayesian
data analysis with R. We are always interested to receive feedback;
positive or negative, and also welcome questions about your own
analyses; feel free to email us.

\hypertarget{refs}{}
\begin{CSLReferences}{1}{0}
\leavevmode\vadjust pre{\hypertarget{ref-O_Neill_2018}{}}%
ONeill, L. G., Parker, T. H., \& Griffith, S. C. (2018). Nest size is
predicted by female identity and the local environment in the blue tit (
cyanistes caeruleus ), but is not related to the nest size of the
genetic or foster mother. \emph{Royal Society Open Science},
\emph{5}(4), 172036. \url{https://doi.org/10.1098/rsos.172036}

\end{CSLReferences}

\backmatter
\end{document}
